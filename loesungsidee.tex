\subsection {Lösungsidee}
	Wir haben uns entschlossen, die Aufgabe mit einer Liste zu lösen: 
	Im ersten Listenelement ist ausschließlich der Ursprungszustand gespeichert. Nun führen wir auf diesen Ursprungszustand folgende Schritte durch:

	\begin{enumerate}
		\item Drehe das Element (der Zustand der Puzzleaufgabe) sowohl nach links als auch nach rechts.
		\item Ist der in 1. erzeugte Zustand neu und keine Lösung der Aufgabe, wird dieser im nachfolgenden Listenelement gespeichert.
	\end{enumerate}

	Daraufhin wird diese Logik für alle Elemente im nachfolgenden Listenelement wiederholt.
	Sobald ein Element aus dem Puzzle fällt, wird die Schleife unterbrochen und die Lösung ausgegeben. Sobald Drehungen der Zustände in dem letzten Listeneintrag keine neuen Zustände liefern, kann mit Sicherheit gesagt werden, dass keine Lösung vorhanden ist.

\subsection{Laufzeit}
	Wenn wir nicht überprüfen würden, ob der neue Zustand der Puzzleaufgabe neu ist, wüchse die Anzahl der Elemente in einem Listeneintrag mit \(2^n\).  
	n entspräche der Anzahl der Iterationen. Da wir allerdings nur neue Objekte aufnehmen, steigt die Anzahl deutlich langsamer.
	Eine genaue Formel für das Ebenenwachstum konnten wir nicht finden, jedoch konnten wir beweisen dass es nur endlich verschiedene Zustände eines Puzzles gibt. Gäbe es unendlich viele Puzzlezustände, würde sich unser Programm nicht beenden. Die Anzahl der möglichen Zustände \(m\) hängt dabei sowohl von der Puzzlegröße als auch von Anzahl und Größe der Stäbchen ab.

	Hat die Puzzleaufgabe keine Lösung, müssen alle \(m\) möglichen Zustände berechnet werden. Nur so kann bewiesen werden, dass es sicher keine Lösung gibt.
	Hat die Puzzleaufgabe hingegen eine Lösung, müsste man nur in einem Worst-Case Szenario ebenfalls alle  Zustände erzeugen, schließlich könnte die Lösung der zuletzt erzeugte Zustand sein. Theoretisch könnte auch der erste Zustand direkt die Lösung sein, die Laufzeit entspräche dann \(O(1)\).
	Die obere Schranke für die Laufzeitkomplexität ist allerdings \(O(m)\). Da uns m nicht bekannt ist, können wir die maximale Laufzeit so jedoch nicht vorhersagen.

	Wir konnten allerdings eine Schranke für das maximal mögliche m ermitteln:

	\[m \le \frac{f!}{(f-s)!}\]

	\(s\) entspricht der Anzahl der Steine, \(f\) der Anzahl der Felder. 

	\(m\) ist allerdings in den meisten Fällen \textbf{deutlich} kleiner als die oben genannte Schranke, da größere Steine für eine geringere Zahl an möglichen Zuständen sorgen. Ferner kann nicht jeder theoretisch mögliche Zustand des Puzzles durch Kippen nach rechts oder links erreicht werden.