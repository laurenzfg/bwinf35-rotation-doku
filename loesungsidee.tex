\subsection {Lösungsidee}
	Zur Lösung des Problems betrachte man mehrere Ebenen. Auf jeder dieser Ebenen lassen sich Zustände der Puzzleaufgabe speichern. 
	Auf Ebene 1 liegt nur der Ausgangszustand. Für jedes Element einer Ebene führe man folgende Logik aus:

	\begin{enumerate}
		\item Drehe das Element (der Zustand der Puzzleaufgabe) sowohl nach links als auch nach rechts.
		\item Ist der in 1. erzeugte Zustand neu und keine Lösung der Aufgabe, wird dieser auf der nachfolgenden Ebene gespeichert.
	\end{enumerate}

	Daraufhin wird diese Logik für alle Elemente auf der nächsten Ebene wiederholt.
	Sobald ein Element nach den Drehungen aus dem Puzzle fällt, wird die Logik unterbrochen und die Lösung ausgegeben. Sobald Drehungen der Zustände einer Ebene ausschließlich bekannte Zustände liefern, kann mit Sicherheit gesagt werden, dass keine Lösung vorhanden ist, da folglich alle durch Drehen erreichbaren Zustände bekannt sind.
	
\subsection{Laufzeit}
	Wenn man nicht überprüfen würde, ob der neue Zustand der Puzzleaufgabe tatsächlich neu ist, würde die Ebenengröße (Anzahl der Elemente auf einer Ebene) mit \(2^n\) wachsen.  
	n wäre dabei die Anzahl der Rotationen. Durch die Überprüfung auf Bekanntheit des Zustands spart man sich sehr viele Elemente. 
	Wie stark man allerdings das Wachstum der Ebenengröße dadurch begrenzt, konnten wir nicht feststellen. Indirekt lässt sich dazu jedoch eins sagen: 
	Eine Puzzleaufgabe kann insgesamt nur eine endliche Zahl \(m\) an Zuständen haben. 

	Hat die Puzzleaufgabe keine Lösung, müsste man mit unserem Algorithmus alle \(m\) möglichen Zustände erzeugen, um dies nachzuweisen.
	Hat die Puzzleaufgabe eine Lösung, müsste man nur in einem Worst-Case Szenario alle  Zustände erzeugen, in allen anderen aber meist deutlich weniger.
	Die obere Schranke für die Laufzeitkomplexität beträgt also \(O(m)\). Da uns diese Anzahl \(m\) der Zustände jedoch nicht bekannt ist, 
	hilft uns diese Überlegung in Bezug auf die Laufzeit nicht weiter. 

	Mit ihr kann man jedoch feststellen, dass unser Algorithmus in jedem Fall in \(m\) Schritten feststellen kann, ob das Problem eine Lösung hat und falls ja, diese auch berechnen. 
	Auch lässt sich für m eine obere Schranke festlegen: \(m \le \frac{f!}{(f-s)!}\) \(s\) ist dabei die Anzahl der Steine und \(f\) die Anzahl der Felder. 
	In dieser Vereinfachung wird angenommen, dass jeder Stein nur die Größe eines Feldes hat. Dies wird in den meisten Fällen jedoch nicht so sein.
	Dazu kommt noch, dass man je nach Ausgangssituation, 
	gar nicht alle theoretisch möglichen Zustände durch eine Verkettung von Rotationen erzeugen kann. 
	Also ist \(m\) in den meisten Fällen deutlich kleiner. 

