	Für das Speichern der Zustände der Puzzleaufgabe nutzen wir einen zweidimensionalen Array. 
	Da die Größe dieses Arrays zum Zeitpunkt der Kompilierung unbekannt ist, nutzen wir pointer-to-pointer.
	Das Hauptverfahren läuft in der Funktion \texttt{void loese(char **f))} der Hauptklasse \texttt{rahmen}.
  Als weitere Hilfsfunktion gibt es \texttt{void rotieren\_links(char **f)}, zum Rotieren nach links, 
	und das enstprechende Gegenstück \texttt{void rotieren\_rechts(char **f)} zum Rotieren nach rechts.
	Innerhalb der die Rotation umsetzenden Funktionen wird am Ende immer noch die Funktion \texttt{void gravitation(char **f)} aufgerufen.
	Diese lässt schwebende Steine nach unten fallen. 
	
	\texttt{bool ist\_loesung(char **f)} überprüft, ob der übergebene Zustand eine Lösung darstellt.
	Zur Umsetzung der Ebenenstruktur nutzen wir einen std::vector \\\texttt{std::vector<std::vector<char **> > baum}.
	In erster Dimension werden in diesem std::vector die Ebenen und in zweiter die Zustände der Puzzleaufgabe gespeichert. 
	Man könnte auch sagen, dass die erste Dimension über die Ebenen und die zweite über die Zustände der Puzzleaufgabe iteriert.

	Um das Überprüfen auf das Vorhandensein eines Zustandes zu beschleunigen, berechnen wir für alle Zustände der Puzzleaufgabe einen Hash.
	Über diesen lassen sich  die Zustände leichter vergleichen. Zum Speichern der Hashwerte nutzen 
	wir einen mehrdimensionalen std::vector \\\texttt{std::vector<std::vector<char *> > hashs}. Dieser hat die gleiche Strukutur wie der std::vector \texttt{baum}.
	
	Mit dem Hashverfahren und einer effizienteren Suchlogik konnte die Laufzeit bei rotation3\_03.txt um den Faktor 3 verringert werden.
 	Die Suchlogik haben wir dahingehend verändert, dass zunächst die jeweils letzten Zustände verglichen werden, da ähnliche Zustände wahrscheinlich weniger Rotationsvorgänge zurückliegen. So können wir schneller neue zustände als schon vorhanden ausschließen.
	
	Die zur Umsetzung der Gravitation genutzte Funktion \texttt{void gravitation(char **f)} geht einen Zustand der Puzzleaufgabe von unten nach oben durch.
	Trifft der Algorithmus auf den Teil eines Steins, wird die Größe dieses Steins bestimmt. Befinden sich unter dem Stein keine Hindernisse, 
	wird dieser ensprechend weit nach unten versetzt. Dabei merkt sich das Programm, welche Steine es bereits versetzt hat. 
	Trifft es auf einen bereits versetzten Stein, wird dieser ignoriert.

 
	
	
