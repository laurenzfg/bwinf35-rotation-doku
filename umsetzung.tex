	Für das Speichern der Zustände der Puzzleaufgabe nutzen wir einen auf dem zweidimensionalen Array. 
	Da die Größe dieses Arrays zum Zeitpunkt der Compilierung unbekannt ist, nutzen wir pointer-to-pointer.
	Das Hauptverfahren läuft in der Funktion \(void\ loese(char\ **f)\) der Hauptklasse \(rahmen\).
  Als weitere Hilfsfunktion gibt es \(void\ rotieren\_links(char\ **\&f)\), zum rotieren nach links, 
	und das enstprechende Gegenstück \(void\ rotieren\_rechts(char\ **\&f)\) zum rotieren nach rechts.
	Innerhalb der rotierenden Funktionen wird am Ende immer noch die Funktion \(void\ gravitation(char\ **\&f)\) aufgerufen,
	diese lässt schwebende Steine nach unten fallen. Um zu überprüfen ob ein Zustand der Puzzleaufgabe eine Lösung ist, 
	dient die Funktion \(bool\ ist\_loesung(char\ **f)\).
	Zur Umsetzung der Ebenen nutzen wir einen std::vector \(std::vector<std::vector<char\ **>\ >\ baum\).
	In erster Dimension werden in diesem std::vector die Ebenen und in Zweiter die Zustände der Puzzleaufgabe gespeichert. 
	Man könnte auch sagen, dass die erste Dimension über die Ebenen und die zweite über die Zustände der Puzzleaufgabe iteriert.
	
	
