\documentclass[parskip=full, DIV=14]{scrartcl}

\usepackage[T1]{fontenc}
\usepackage{selinput}% Eingabecodierung automatisch ermitteln siehe <http://ctan.org/pkg/selinput>
\SelectInputMappings{
	adieresis={ä},
	germandbls={ß},
} 
\usepackage{ngerman}

\usepackage{amsmath}

\usepackage{graphicx}
\graphicspath{{Bilder/}}

\usepackage{listings}
\usepackage[dvipsnames]{xcolor}

\lstset{language={C++},
        numbers=left,
        stepnumber=1,
        numbersep=5pt,
        numberstyle=\tiny,
        breaklines=true,
        breakautoindent=true,
        postbreak=\space,
        tabsize=2,
        basicstyle=\ttfamily\footnotesize,
        showspaces=false,
        showstringspaces=false,
        extendedchars=true,
        commentstyle=\color{Gray}, % comment color
    	keywordstyle=\color{Bittersweet}, % keyword color
    	stringstyle=\color{Orange}, % string color
        % Sorgt dafür, dass das Paket listings auch mit den Sonderzeichen in UTF-8 zurecht kommt.
				literate=
					{Ö}{{\"O}}1
					{Ä}{{\"A}}1
					{Ü}{{\"U}}1
					{ß}{{\ss}}2
					{ü}{{\"u}}1
					{ä}{{\"a}}1
					{ö}{{\"o}}1
        }

\usepackage{scrpage2}
\usepackage[hidelinks]{hyperref}

\newcommand{\shellcmd}[1]{\texttt{\$ #1}\\}
\newcommand{\shellout}[1]{\texttt{#1}\\}

\begin{document}
	\pagestyle{scrheadings}

	\ihead{Aufgabe 3: Rotation, Team 00001}\ohead{Jon Fehling, Laurenz Grote}

	\titlehead{Aufgabe 3: Rotation, Team 00001 \hfill Jon Fehling, Laurenz Grote}
	\title{Rotation}
	\subtitle{Aufgabe 3}
	\author{Jon Amos Fehling \& Laurenz Friedrich Grote}
	\date{}
	\maketitle
	\tableofcontents
	
	\vspace {2em}
	Unsere Umsetzung für "`Rotation"' erfolgte unter Ubuntu mit C++/C (Compiler: g++ 5.4.0). Das Programm liegt als Quellcode vor.
	\clearpage
	% ----------------------------------------------------------------------------
	\section{Lösungsidee}
		\subsection {Lösungsidee}
	Zur Lösung des Problems betrachte man mehrere Ebenen. Auf jeder dieser Ebenen lassen sich Zustände der Puzzleaufgabe speichern.
	Auf Ebene 1 liegt nur der Ausgangszustand. Für jedes Element einer Ebene führe man folgende Logik aus:

	\begin{enumerate}
		\item Drehe das Element (der Zustand der Puzzleaufgabe) sowohl nach links als auch nach rechts.
		\item Ist der in 1. erzeugte Zustand neu und keine Lösung der Aufgabe, wird dieser auf der nachfolgenden Ebene gespeichert.
	\end{enumerate}

	Daraufhin wird diese Logik für alle Elemente auf der nächsten Ebene wiederholt.
	Sobald ein Element nach den Drehungen aus dem Puzzle fällt, wird die Logik unterbrochen und die Lösung ausgegeben. Sobald Drehungen der Zustände einer Ebene ausschließlich bekannte Zustände liefern, kann mit Sicherheit gesagt werden, dass keine Lösung vorhanden ist, da folglich alle durch Drehen erreichbaren Zustände bekannt sind.

\subsection{Laufzeit}
	Wenn man nicht überprüfen würde, ob der neue Zustand der Puzzleaufgabe tatsächlich neu ist, würde die Ebenengröße (Anzahl der Elemente auf einer Ebene) mit \(2^n\) wachsen.
	n wäre dabei die Anzahl der Rotationen. Da wir aber nur unbekannte Zustände aufnehmen, ist das Ebenenwachstum deutlich geringer.
	Wie stark man allerdings das Wachstum der Ebenengröße dadurch begrenzt, konnten wir nicht feststellen. Jedoch lässt sich dazu sagen:
	Eine Puzzleaufgabe kann insgesamt nur eine endliche Zahl \(m\) an Zuständen haben.

	Hat die Puzzleaufgabe keine Lösung, müsste man mit unserem Algorithmus alle \(m\) möglichen Zustände erzeugen, um dies nachzuweisen.
	Hat die Puzzleaufgabe eine Lösung, müsste man nur in einem Worst-Case Szenario alle  Zustände erzeugen, in allen anderen aber meist deutlich weniger.
	Die obere Schranke für die Laufzeitkomplexität beträgt also \(O(m)\). Da uns diese Anzahl \(m\) der Zustände nicht bekannt ist,
	hilft diese Überlegung in Bezug auf die Laufzeit nicht weiter.

	Mit ihr kann man jedoch feststellen, dass unser Algorithmus in jedem Fall in \(m\) Schritten feststellen kann, ob das Problem eine Lösung hat und falls ja, diese auch berechnen.
	Auch lässt sich für m eine obere Schranke festlegen: \(m \le \frac{f!}{(f-s)!}\) \(s\) ist dabei die Anzahl der Steine und \(f\) die Anzahl der Felder.
	In dieser Vereinfachung wird angenommen, dass jeder Stein nur die Größe eines Feldes hat. Dies wird in den meisten Fällen jedoch nicht so sein.
	\(m\) ist allerdings in den meisten Fällen deutlich kleiner als die oben genannte Schranke, da größere Steine für eine geringere Zahl an möglichen Zuständen sorgen.
	Ferner kann nicht jeder theoretisch mögliche Zustand des Puzzles durch Rotationen erreicht werden.

	\clearpage
	\section{Umsetzung}
			Für das Speichern der Zustände der Puzzleaufgabe nutzen wir einen zweidimensionalen Array.
	Da die Größe dieses Arrays zum Zeitpunkt der Kompilierung unbekannt ist, nutzen wir pointer-to-pointer.
	Das Verfahren läuft in der Funktion \texttt{void loese(char **f))} der Hauptklasse \texttt{rahmen}.
	Als weitere Hilfsfunktion gibt es \texttt{void rotieren\_links(char **f)} sowie \texttt{void rotieren\_rechts(char **f)}. Diese setzen die Rotation eines Rahmens nach links oder rechtes um.
	Innerhalb der beiden Funktionen wird auf die Funktion \texttt{void gravitation(char **f)} zurückgegriffen.
	Diese simuliert den Fall der Steine im vorher gedrehten Puzzle.
	
	\texttt{bool ist\_loesung(char **f)} überprüft, ob der übergebene Zustand eine Lösung darstellt.
	
	Zur Umsetzung der Ebenenstruktur nutzen wir einen std::vector \\\texttt{std::vector<std::vector<char **> > baum}.
	In erster Dimension werden in diesem std::vector die Ebenentiefen und in zweiter die jeweiligen Zustände der Puzzleaufgabe gespeichert. 
	Auf die Aufgabenstellung angewandt bedeutet dies, dass die erste Vektor über die Anzahl der Drehungen iteriert, während die inneren Vektoren die jeweils möglichen Zustände beinhalten.

	Um das Überprüfen auf das Vorhandensein eines Zustandes zu beschleunigen, berechnen wir für alle Zustände der Puzzleaufgabe einen Hash.
	Über diesen lassen sich  die Zustände leichter vergleichen. Zum Speichern der Hashwerte nutzen 
	wir einen mehrdimensionalen std::vector \\\texttt{std::vector<std::vector<char *> > hashs}. Dieser hat die gleiche Strukutur wie der std::vector \texttt{baum}.
	
	Mit dem Hashverfahren und einer effizienteren Suchlogik konnte die Laufzeit bei rotation3\_03.txt um den Faktor 3 verringert werden.
 	Die Suchlogik haben wir dahingehend verändert, dass zunächst die jeweils letzten Zustände verglichen werden, da ähnliche Zustände wahrscheinlich weniger Rotationsvorgänge zurückliegen. So können wir schneller neue zustände als schon vorhanden ausschließen.

	Die zur Umsetzung der Gravitation genutzte Funktion \texttt{void gravitation(char **f)} geht einen Zustand der Puzzleaufgabe von unten nach oben durch.
	Trifft der Algorithmus auf den Teil eines Steins, wird die Größe dieses Steins bestimmt. Befinden sich unter dem Stein keine Hindernisse, 
	wird dieser ensprechend weit nach unten versetzt. Dabei merkt sich das Programm, welche Steine es bereits versetzt hat, sodass bereits versetzte Steine ignoriert werden. So werden Endlosschleifen verhindert.
 
	
	

	\clearpage
	\section{Beispiel}
		Der Compilierungsbefehl lautet:
\shellcmd{g++ rahmen.cpp rahmen.h main.cpp -o rotation.out -std=c++11}

Alle drei gegebenen Beispiele hatten eine Lösung.
Für alle Beispiele sind die Input- und vollständigen Outputdateien im Ordner ... zu finden.
\subsection{Beispiel 1}
	Gekürzte Programmausgabe:\\
	\shellout{  \$ ./rotation.out \\
				Dateiname: rotation1\_03.txt   \\
				Beste Loesung mit Laenge 6 : llllrl\\
				Laufzeit: 0.000822s\\
}
\subsection{Beispiel 2}
	Gekürzte Programmausgabe:\\
	\shellout{  \$ ./rotation.out \\
				Dateiname: rotation2\_03.txt \\
				Beste Loesung mit Laenge 22 : rrrrlrrlllllllrrrrrrlr \\
				Laufzeit: 0.004817s
}
\subsection{Beispiel 3}
	Gekürzte Programmausgabe:\\
	\shellout{ \$ ./rotation.out 
				Dateiname: rotation3\_03.txt\\
				Beste Loesung mit Laenge 90 : lrrrrlllrrrlllllllrrllrrrrrrrlrrrrrrrrrrrlllrrrrrlllll
				rrrrllrrrrrrrlrlllllllllllrllrrrrrlr\\
				Laufzeit: 956.263s
}
\clearpage
\subsection{Beispiel 4}
	Hier noch ein Beispiel, für das keine Lösung existiert.\\
	Gekürzte Programmausgabe:\\
	\shellout{ \$ ./rotation.out \\ 
								Dateiname: rotation4\_03.txt \\
								Keine Loesung vorhanden. \\
								Laufzeit: 0.036352s
}



	\clearpage
	\section{Quellcode}
		\subsection{Lösen}
		\lstinputlisting[firstline=296,lastline=384]{code/rahmen.cpp}
\subsection{Rotieren links}
		\lstinputlisting[firstline=99,lastline=110]{code/rahmen.cpp}
\clearpage
\subsection{Rotieren rechts}
		\lstinputlisting[firstline=112,lastline=123]{code/rahmen.cpp}
\subsection{Steine nach unten fallen lassen}
		\lstinputlisting[firstline=125,lastline=255]{code/rahmen.cpp}
\clearpage
\subsection{Lösungsüberprüfung}
		\lstinputlisting[firstline=257,lastline=293]{code/rahmen.cpp}

\end{document}
